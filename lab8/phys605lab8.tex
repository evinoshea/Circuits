
\documentclass[twocolumn, amsmath]{revtex4}

\usepackage{graphicx}
%\graphicspath{ {tex_pics/} }


\begin{document}


\title{PHYS 605 Lab \#8} 

\author{Morgan A. Daly}
\author{Evin O'Shea}
\date{\today} 


\maketitle


\section{Introduction and Theory}
\subsection{Purpose}
intro:
This lab focused on working with digital ciurcuits. The two digital compontents used were the 555 timer and the nand gate. 

The goal of the first part of the lab was to demonstrate how nand gates work and how they can be used in various ways to create more interesting opertations by combining nand gates. 
the second part of the lab focusxed on the 555 timer. for this part of the lab, they design of the circuit would give a visual queue depending on the output of the 555 timer to demonstrate what the outputwas.



\subsection{Background / Theory}


BG/theory 
Digital circuits consist of different high square pulses that can be catagorized as one of two states: high or low; 1 or 0. This type of circuitry is used in a lot of technology because it can send encoded data without error. The error is reduced, because a low andf high pulse can be easily distinguisehed even if there is noise in the signal. THis means that the data is only limited by the number of pulses sent.

In the lab, the gropup used LED's to get a visual feedback on the output of the circuit. 
LED's require a minimum voltage across them to light up. This is perfect for digital circuits as the high voltage is over the diode voltage and the low is below it. This means that the LED on is a high coming through and the LED off is as low comming through.

\begin{figure}[h]
    \includegraphics[scale=0.3]{something.png}  
    \caption{}
\end{figure}


\section{Methodology}

\begin{enumerate}
    \item Connect the oscilloscope directly to the ProtoBoard's AC voltage supply and measure the output voltage.
    \item Set up the circuit shown in figure (1) record the voltage across the resistor and record the resistor value
    \item Build the op amp circuit from figure (2).
    \item Take record of the output voltage with no load attached.
    \item Add a load to the circuit with a large value as to eliminate sag.
    \item Measure the voltage across the load resistor and the value of this resistor.
    \item Change the frequency and take nots on any changes in the plot on the oscilloscope.
    \item Change the load resistor and investigate the limitations of the circuit.
    \item Build the circuit from figure (3).
    \item Take recordings of output voltage as the frequency is varied and the input voltage is kept constant.
\end{enumerate}


\section{Results and Analysis}

\subsection{Data}
For part A of the lab, the only data to be recorded was from the visual feedback of the LED's. The logic table shown below demonstrates the lows and highs at various points in the circuit:

\begin{center}
	\begin{ruledtabular}
    \begin{tabular}{ l l l l l l l}
	S1 & S2 & 1,2 & 5,6 & 3,8 & 4,9 & 10, LED \\ \colrule
	0 & 0 & 0 & 0 & 0 & 0 & 0  \\
	0 & 1 & 0 & 1 & 1 & 0 & 1 \\
	1 & 0 & 1 & 0 & 0 & 1 & 1  \\
	1 & 1 & 1 & 1 & 0 & 0 & 1 \\
\end{tabular}
    \end{ruledtabular}
\end{center}

For part B of the lab, the resistors and capacitor values used in the 555 timer circuit affected the frequency of the output. The values of the resistors and capacitors used were:

R$_a$ = 677 k$\Omega$

R$_b$ = 35.18 k$\Omega$

C = 0.947 $\mu$F

The output frequency recorded from the output of the 555 timer was:

f$_{out}$ = 1.992 Hz

\subsection{Calculations}
For part A of the lab, there are no calculations to be made. All of the circuitry is logically operated. The lab only had the goal of obtaining the desired output.

For part B of the lab, the expected output frequency could be calculated. %HERE NEED TO Know which one is for this section
Using equation (1), the expected output of the 555 timer was:

f$_{expected}$ = $\frac{1.4}{C(2R_a + R_b)}$ = $\frac{1.4}{0.947x10^{-6}(2(677x10^3) + 35.18x10^3}$ = 1.98 Hz

This means that the percent error for expected versus actual output frequency can be calculated. This is shown below:

\%error = $\frac{f_{out} - f_{expected}}{f_{expected}}$x100 = $\frac{1.992 - 1.98}{1.98}$x100 = 0.704\%

\subsection{Analysis}
For part A of the lab, the logic gates operated as expected. The working of the circuit was explained in the background. Since the circuit was digital, the output does not have error, only correct or incorrect output.

For part B of the lab, the only data collected was the output frequency of the 555 timer. The output frequency only varied from the expected frequency by 0.704\%. Since the percent error of this was very low, it means the measurements were done accurately and the circuit was built correctly.





\section{Conclusion}
The goal of the first part of the lab was completed successfully. The desired output was obtained and checked during the lab.

For the second part of the lab, the only goal was to obtain a frequency of about 2 Hz. This was achieved with low percent error. This part of the lab was also successfully completed.

\end{document}

