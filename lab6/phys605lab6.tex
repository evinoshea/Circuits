
\documentclass[twocolumn, amsmath]{revtex4}

\usepackage{graphicx}
%\graphicspath{ {tex_pics/} }


\begin{document}


\title{PHYS 605 Lab \#6} 

\author{Morgan A. Daly}
\author{Evin O'Shea}
\date{\today} 


\maketitle


\section{Introduction and Theory}
\subsection{Purpose}

In this lab, the focus was to learn about how diodes can be used in AC circuits. The first part of the lab showed how a single diode can be used as a half-wave rectifier. In the second part of the lab the group made a full wave rectifier. In the third part of the lab the behavior of a zener diode was explored both with DC and AC voltage sources.

\subsection{Background / Theory}

The lab revolved around the behavior of diodes in AC circuits. Diodes are an interesting type of passive circuit element which have interesting properties. The most important property of diodes is that they only allow current flow in one direction. When an AC voltage source powers a diode, only the positive or negative voltage will go through the diode. 
%not sure where to put this if at all
%The second interesting property of diodes is that they have a minimum voltage required to pass current through/ has a max? clipps the voltage?

In the Part A of the lab a diode was used to convert the AC signal into only a positive signal. This demonstrated the diode property of only allowing current flow in one direction. When the AC source supplies a positive voltage the diode would allow current flow through. When the voltage source supplies a negative voltage the diode will not let current flow through. The circuit used for this part of the lab was built in two aprts. First the circuit was built with a resistor and a diode in series with the oscilliscope connected in parallel with the voltage source and the diode. After this circuit was investigated a 3V DC supply was added to the circuit between the resistor and diode. The final circuit for this part of the lab is shown below:

\begin{figure}
    \includegraphics[scale=0.25]{halfwave.png}  
    \caption{This circuit only supplies postive voltage to the oscilloscope.}
\end{figure}

In the second part of the lab, the diode was used to allow the positive and negative voltages through while inverting the negative voltage. The circuit used for this part of the lab is shown below:

\begin{figure}
    \includegraphics[scale=0.3]{bridge.png}  
    \caption{A full wave rectifier with input and output connected to an ocsilloscope}
\end{figure}

The circuit makes all the voltage positive. As positive voltage is supplied to the circuit current will pass through the top left diode and not the bottom left because of the orientation of the diode. Current will then flow down through the resistor and not the top right diode. As negative voltage is supplied to the circuit current will flow from the bottom side of the AC source. As the current flows to the right side of the bridge it will pass through the top right diode and not the bottom right. The current will then flow down through the resistor as it did when a positive voltage was supplied. This combination will cause flow in only one direction through the resistor and the output of the bridge.

For the third part of the lab, a zener diode was investigated. The distinction between a regular diode and a zener diode is that a zener diode will allow voltage to pass in both directions, but in the reverse-bias direction, there is a minimum voltage required to cause current flow. This voltage is called the zener voltage. A diagram of the circuit for this part of the lab is shown below:

\begin{figure}
    \includegraphics[scale=0.3]{zener.png}  
    \caption{This circuit only supplies postive voltage to the oscilloscope.}
\end{figure}

The left side was connected to both DC and AC sources to invesitage different properties of the zener diode. 
%One important use of the zener diode was to "clip" the voltage source. This means that the zener diode caps the voltage allowed through... 
%********need to add more when I can see plots******





\section{Methodology}

\begin{enumerate}
    \item Construct RC circuit with oscilloscope as show in figure (1) without the 3V DC source and diode attatched.
    \item Take record of the plot from the oscilloscpe.
    \item Adjust frequency and amplitude of input source and repeat step 2.
    \item Build diode brdge shown in figure (2).
    \item Connect input source (one that is external from the protoboard) and connect the oscilloscope as shown in figure (2).
    \item Repeat steps 2 and 3 to get data about the bridge.
    \item Construct the circuit shown in figure (3).
    \item Add an adjustable DC input to the left side of the circuit and connect a measurement device to the right side of the circuit.
    \item Make recordings of the output voltages as the input voltage is modified.
    \item Swap the DC input for an AC input and swap the measurement device for one suited for AC voltages (oscilloscope) if necessary.
    \item Take record of the plot shown on the oscilloscope.
    \item Swap the zener diode out for another one and repeat step 11.
    \item Combine the two diodes in series in the same direction and make record of the plot on the oscilloscope.
    \item Combine the two zener diodes in parallel in opposite directions to "clip" both positive and negative voltages from the AC input.
\end{enumerate}


\section{Results and Analysis}

\subsection{Data}
For part A of the lab, first only part of the circuit was built as described previously. This setup was investigated with varying voltages and frequencies for the input source. All voltages for $V_{in}$ and $V_{out}$ were maximum voltages.


The input source was initially set to an amplitude and frequency of 2.64V and 7.225Hz respectively. The $V_{out}$ for the diode was 1.80V.

\begin{figure}
    \includegraphics[scale=0.3]{1800mV.png}  
    \caption{$V_{in}$ = 2.64V f=7.225Hz $V_{out}$= 1.80V}
\end{figure}


The group then increased the amplitude of the input voltage to 4.08V. The resulting $V_{out}$ was 2.72V.

\begin{figure}
    \includegraphics[scale=0.3]{2720mV.png}  
    \caption{$V_{in}$ = 4.08V f=7.225Hz $V_{out}$= 2.72V}
\end{figure}


The group then reset the input voltage to 2.64V and increased the frequency to 75.19Hz. The resulting $V_{out}$ was 1.76V

\begin{figure}
    \includegraphics[scale=0.3]{1760mV.png}  
    \caption{$V_{in}$ = 2.64V f=75.19Hz $V_{out}$= 1.76V}
\end{figure}

%fourth frequency: same for higher

After adjusting the amplitude and frequency of the input voltage a 3.065V DC source was added to the circuit attatched by another diode as shown in figure (1). 

The group then decreased the voltage of the input source to 1.48V and set the frequency to 7.225Hz as it was initially. The resulting $V_{out}$ was 460mV

\begin{figure}
    \includegraphics[scale=0.3]{460mV.png}  
    \caption{$V_{in}$ = 1.48V f=7.225Hz $V_{out}$= 460mV}
\end{figure}

The input voltage and frequency were then set to 2.64V and 731.0mHz respectively. The $V_{out}$ for the diode was 464mV.

\begin{figure}
    \includegraphics[scale=0.3]{464mV.png}  
    \caption{$V_{in}$ = 2.64V f=731.0mHz $V_{out}$= 464mV}
\end{figure}


The group increased the frequency to 74.63Hz. The resulting $V_{out}$ was 480mV

\begin{figure}
    \includegraphics[scale=0.3]{480mV.png}  
    \caption{$V_{in}$ = 2.64V f=74.63Hz $V_{out}$= 480mV}
\end{figure}




%don't need this right??
%\subsection{Calculations}



 
\subsection{Analysis}
For part A of the lab the voltage across a single diode was measured with varying input voltages. When only an AC voltage source was connected to the diode and a resistor that were in series, the result was half of a sine wave as expected because the diode only allows current flow in one direction. The rest of the curve was a line with a small positive slope. 
%THIS I'm not sure about annnd
This is also because the initial part of the positive input will not cause current flow as the diode has a minimum voltage required to cause current flow. 

When the amplitude of the input source was increased, the straight line looked more flat because it was the same size as before, but the amplitude of the positive curve was not larger. The other part of the curve simply had a larger amplitude because the input was larger. 

As the frequency was increased, the linear part of the curve looked more flat and the amplitude decreased only slightly. This happened because the minimum voltage for the diode was reached faster. As the frequency was further increased, the plot looked more like a DC source that was turning on and off.
%and again not fully sure....

After the 3.065V DC source was added with the diode as shown in figure (1). The amplitude of the oscillations of voltage across the diode were now flat on top and bottom. The added voltage caused the voltage to be capped as the input voltage increased. The bottom was again flat because the negative part of the AC source would not pass through the diode.
%not sure about the curvyness again...

When the frequency was increased, the plot flattened out on top and bottom and looked again like a DC input that was flicketing up and down by 480mV. This makes sense as the diode will reach the maximum voltage it can allow quickly and will stay  constant until the AC source supplies a negative voltage which will not affect the voltage across the diode.

%When the frequency was decreased and the voltage was increased, the plot looekd very peculiar.  
% D:





\section{Conclusion}
The first part of the lab was completed successfully as the group was able to demonstrate that a diode can be used to create close to half of a DC voltage supply. The group discovered that for higher frequency input voltages the plot was more flat on top and bottom. The correct properties of the diode were deomonstrated in this part of the lab.


\end{document}

